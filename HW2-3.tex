\documentclass{article}\usepackage[]{graphicx}\usepackage[]{color}
% maxwidth is the original width if it is less than linewidth
% otherwise use linewidth (to make sure the graphics do not exceed the margin)
\makeatletter
\def\maxwidth{ %
  \ifdim\Gin@nat@width>\linewidth
    \linewidth
  \else
    \Gin@nat@width
  \fi
}
\makeatother

\definecolor{fgcolor}{rgb}{0.345, 0.345, 0.345}
\newcommand{\hlnum}[1]{\textcolor[rgb]{0.686,0.059,0.569}{#1}}%
\newcommand{\hlstr}[1]{\textcolor[rgb]{0.192,0.494,0.8}{#1}}%
\newcommand{\hlcom}[1]{\textcolor[rgb]{0.678,0.584,0.686}{\textit{#1}}}%
\newcommand{\hlopt}[1]{\textcolor[rgb]{0,0,0}{#1}}%
\newcommand{\hlstd}[1]{\textcolor[rgb]{0.345,0.345,0.345}{#1}}%
\newcommand{\hlkwa}[1]{\textcolor[rgb]{0.161,0.373,0.58}{\textbf{#1}}}%
\newcommand{\hlkwb}[1]{\textcolor[rgb]{0.69,0.353,0.396}{#1}}%
\newcommand{\hlkwc}[1]{\textcolor[rgb]{0.333,0.667,0.333}{#1}}%
\newcommand{\hlkwd}[1]{\textcolor[rgb]{0.737,0.353,0.396}{\textbf{#1}}}%
\let\hlipl\hlkwb

\usepackage{framed}
\makeatletter
\newenvironment{kframe}{%
 \def\at@end@of@kframe{}%
 \ifinner\ifhmode%
  \def\at@end@of@kframe{\end{minipage}}%
  \begin{minipage}{\columnwidth}%
 \fi\fi%
 \def\FrameCommand##1{\hskip\@totalleftmargin \hskip-\fboxsep
 \colorbox{shadecolor}{##1}\hskip-\fboxsep
     % There is no \\@totalrightmargin, so:
     \hskip-\linewidth \hskip-\@totalleftmargin \hskip\columnwidth}%
 \MakeFramed {\advance\hsize-\width
   \@totalleftmargin\z@ \linewidth\hsize
   \@setminipage}}%
 {\par\unskip\endMakeFramed%
 \at@end@of@kframe}
\makeatother

\definecolor{shadecolor}{rgb}{.97, .97, .97}
\definecolor{messagecolor}{rgb}{0, 0, 0}
\definecolor{warningcolor}{rgb}{1, 0, 1}
\definecolor{errorcolor}{rgb}{1, 0, 0}
\newenvironment{knitrout}{}{} % an empty environment to be redefined in TeX

\usepackage{alltt}
\usepackage{amsmath} 
\usepackage{amsfonts}
\usepackage{graphicx}
\usepackage{hyperref}
\hypersetup{colorlinks = true,citecolor=black}
\usepackage{natbib}
\bibliographystyle{apalike}
\usepackage[margin=1.0in]{geometry}
\usepackage{float}
\IfFileExists{upquote.sty}{\usepackage{upquote}}{}
\begin{document}
\noindent \textbf{MA 354: Data Analysis -- Fall 2021 -- Due 10/8 at 5p}\\%\\ gives you a new line
\noindent \textbf{Homework 2: Question 4}\vspace{1em}\\
Select a discrete distribution (not the Poisson). It does not 
  have to be one that we cover in the notes! To explore the PMF of your distribution, 
  specify two sets of parameter(s) for your distribution.
  \begin{enumerate}
  %%%%%%%%%%%%%%%%%%%%%%%%%%%%%%%%%%%%%%%%%%%%%%%%%%%%%%%%%%%%%%%%%%%%%%%%%%%%%%%%%%%%%%%%%%%
  %%%%%%%%%  Part (a)
  %%%%%%%%%%%%%%%%%%%%%%%%%%%%%%%%%%%%%%%%%%%%%%%%%%%%%%%%%%%%%%%%%%%%%%%%%%%%%%%%%%%%%%%%%%%
  \item \textbf{History} Discuss what types of random variables are modeled with 
  your distribution. Be sure to include a discussion about the support and ensure
  to provide the mass function, and CDF. This requires some internet research -- 
  what's the history of the distribution, why was it created and named? What are
  some exciting applications of this distribution? Cite all of your sources.\\
  
  \textbf{Solution:}
  The Bernoulli distribution, discussed in both \cite{chap6} and \cite{Bern}, is a special case of the binomial distribution, in which each observation can take one of two outcomes.  We define these as a ``success," where the observation of interest occurs, or a ``failure," where it does not.  Thus, the support for this distribution (or ``the set of all possible values of our random variable") is $\chi = \{x \in \{0,1\}\}$ (Cipolli, 163).  We use p to denote the probability that a ``success" occurs and $q=1-p$ to denote the probability that a ``failure" occurs. Thus, the probability mass function for the Bernoulli distribution is given by $f_X(x|p)=p^x(1-p)^{1-x}$, where x is either 0 or 1 (Sinharay, 132).  We can write this to be robust to other inputs by including an indicator function, I, such that the PMF is given by $f_X(x|p)=p^x(1-p)^{1-x}I(x\in\{0,1\})$ (Cipolli, 173). The cumulative distribution function of the Bernoulli distribution is given by:
$$ F_X(x|p) = 
\begin{cases} 
      0 & x < 0 \\
      1-p & 0\leq x <1 \\
      1 & 1\leq x 
   \end{cases}
$$
(Sinharay, 132).  We could similarly write this using the floor operator and indicator function as:
$$F_X(x|p) = [(1-p)I(\lfloor x \rfloor = 0)]+I(\lfloor x \rfloor \geq 1)
$$
(Cipolli, 173). The Bernoulli distribution, created by Jakob Bernoulli, is often used to in the context of flipping coins(Sinharay, 132).  A fair coin would be represented by a Bernoulli distribution with both $p$ and $q$ equal to 0.5 (while any other parameter values would represent an unfair coin flip).  I will investigate when the parameters p=0.5, q=0.5 and p=0.7,q=0.3.

  %%%%%%%%%%%%%%%%%%%%%%%%%%%%%%%%%%%%%%%%%%%%%%%%%%%%%%%%%%%%%%%%%%%%%%%%%%%%%%%%%%%%%%%%%%%
  %%%%%%%%%  Part (b)
  %%%%%%%%%%%%%%%%%%%%%%%%%%%%%%%%%%%%%%%%%%%%%%%%%%%%%%%%%%%%%%%%%%%%%%%%%%%%%%%%%%%%%%%%%%%
	\item Show that you have a valid PMF. You can show this approximately by 
	calculating the series in a repeat loop until probability mass evaluations are 
	infinitesimally small.\\
  
  \textbf{Solution:}
  For a PMF to be valid, all values of the PMF must remain within 0 and 1, and the sum of all PMF values over the support must be equal to one:
$$ 0\leq f_X(x)\leq 1 \hspace{.5cm} \text{for all } x \in \mathbb{R}$$
$$\sum_{-\infty}^{\infty} f_X(x)=\sum_{\chi} f_X(x)=1$$
(Cipolli, 167).  For a distribution like the Binomial distribution, for example, proving a PMF to be valid could require calculating it for increasing x until the probability is sufficiently small.  However, for the Bernoulli distribution, this is a little more direct because the support includes only two possible observations.  Thus, for each $p\in (0,1)$ for this distribution, the PMF will take only one of two forms:
$f_X(0|p)=p^0(1-p)^{1-0}=1-p$ or $f_X(1|p)=p^1(1-p)^{1-1}=p$.  Because $p\in (0,1)$, each of these possible values for the PMF will remain between 0 and 1, satisfying the first criteria.  Further, because $(1-p)+p=1$ the sum of these is always 1, satisfying the second criteria.
	%%%%%%%%%%%%%%%%%%%%%%%%%%%%%%%%%%%%%%%%%%%%%%%%%%%%%%%%%%%%%%%%%%%%%%%%%%%%%%%%%%%%%%%%%%%
  %%%%%%%%%  Part (c)
  %%%%%%%%%%%%%%%%%%%%%%%%%%%%%%%%%%%%%%%%%%%%%%%%%%%%%%%%%%%%%%%%%%%%%%%%%%%%%%%%%%%%%%%%%%%
	\item Find the median for your two sets of parameter(s). Conduct some research 
	to find the median based on our PMF to confirm that your numerical approach is
	correct.\\
  
  \textbf{Solution:}
  When p=0.5 and q=0.5, the median value depends in some ways on how we define some of the specifics of a median.  Because this situation represents a distribution where an equal number of ``successes" and ``failures" occur, we could say that there is no median.  We could also define the median as the in-between value $x=0.5$.
  When p=0.7 and q=0.3, the median value is more straightforward.  This distribution represents the case when 70\% of observations are ``successes" and 30\% are ``failures," and therefore the median value would be a ``success" with x=1.
  In general for the Bernoulli distribution, the median would be equal to the x that maximizes the PMF.  Said another way, when $p>q$, the median is 1, and when $p<q$, the median is 0.  The border case when $p=q$ and $f_X(0|p)=f_X(1|p)$, as discussed above, depends on how we want to define the nuances of the median in our given context (Sinharay, 132).
	%%%%%%%%%%%%%%%%%%%%%%%%%%%%%%%%%%%%%%%%%%%%%%%%%%%%%%%%%%%%%%%%%%%%%%%%%%%%%%%%%%%%%%%%%%%
  %%%%%%%%%  Part (d)
  %%%%%%%%%%%%%%%%%%%%%%%%%%%%%%%%%%%%%%%%%%%%%%%%%%%%%%%%%%%%%%%%%%%%%%%%%%%%%%%%%%%%%%%%%%%
	\item \label{q3PMF} Graph the PMF for several values of the parameter(s) 
	including the two sets you specified. What does changing the parameter(s) do 
	to the shape of the PMF?\\
	\textbf{Solution:}
\begin{knitrout}
\definecolor{shadecolor}{rgb}{0.969, 0.969, 0.969}\color{fgcolor}\begin{kframe}
\begin{alltt}
\hlcom{##############################################################}
\hlcom{##    ADAPTED FROM CHAPTER 6 NOTES CODE: ch6-Bernoulli.R}
\hlcom{##############################################################}

\hlkwd{library}\hlstd{(}\hlstr{"tidyverse"}\hlstd{)}
\end{alltt}


{\ttfamily\noindent\itshape\color{messagecolor}{\#\# -- Attaching packages --------------------------------------- tidyverse 1.3.1 --}}

{\ttfamily\noindent\itshape\color{messagecolor}{\#\# v ggplot2 3.3.3 \ \ \ \ v purrr \ \ 0.3.4\\\#\# v tibble \ 3.1.1 \ \ \ \ v dplyr \ \ 1.0.6\\\#\# v tidyr \ \ 1.1.3 \ \ \ \ v stringr 1.4.0\\\#\# v readr \ \ 1.4.0 \ \ \ \ v forcats 0.5.1}}

{\ttfamily\noindent\itshape\color{messagecolor}{\#\# -- Conflicts ------------------------------------------ tidyverse\_conflicts() --\\\#\# x dplyr::filter() masks stats::filter()\\\#\# x dplyr::lag() \ \ \ masks stats::lag()}}\begin{alltt}
\hlkwd{library}\hlstd{(}\hlstr{"ggplot2"}\hlstd{)}
\hlkwd{library}\hlstd{(}\hlstr{"patchwork"}\hlstd{)}

\hlcom{#############################################################}
\hlcom{###Bernoulli Distribution ###################################}
\hlcom{#############################################################}
\hlstd{dbern}\hlkwb{<-}\hlkwa{function}\hlstd{(}\hlkwc{x}\hlstd{,}\hlkwc{prob}\hlstd{)\{}                \hlcom{#DEFINING THE PMF}
  \hlkwa{if}\hlstd{(prob}\hlopt{<}\hlnum{0} \hlopt{|} \hlstd{prob}\hlopt{>}\hlnum{1}\hlstd{)\{}
    \hlstd{errormsg} \hlkwb{<-} \hlstr{"This function is only valid for success probabilities between 0 and 1."}
    \hlkwd{stop}\hlstd{(errormsg)}
  \hlstd{\}}
  \hlstd{indicator} \hlkwb{<-} \hlkwd{rep}\hlstd{(}\hlnum{0}\hlstd{,} \hlkwd{length}\hlstd{(x))}
  \hlstd{indicator[x}\hlopt{==}\hlnum{0}\hlstd{]} \hlkwb{<-} \hlnum{1} \hlcom{# indicator should be one if x=0}
  \hlstd{indicator[x}\hlopt{==}\hlnum{1}\hlstd{]} \hlkwb{<-} \hlnum{1} \hlcom{# indicator should be one if x=1}
  \hlstd{fx} \hlkwb{<-} \hlstd{(prob}\hlopt{^}\hlstd{x} \hlopt{*} \hlstd{(}\hlnum{1}\hlopt{-}\hlstd{prob)}\hlopt{^}\hlstd{(}\hlnum{1}\hlopt{-}\hlstd{x))} \hlopt{*} \hlstd{indicator} \hlcom{# PMF formula}
  \hlkwd{return}\hlstd{(fx)}
\hlstd{\}}

\hlstd{pbern}\hlkwb{<-}\hlkwa{function}\hlstd{(}\hlkwc{q}\hlstd{,} \hlkwc{prob}\hlstd{)\{}
  \hlkwa{if}\hlstd{(prob}\hlopt{<}\hlnum{0} \hlopt{|} \hlstd{prob}\hlopt{>}\hlnum{1}\hlstd{)\{}
    \hlstd{errormsg}\hlkwb{<-}\hlstr{"This function is only valid for success probabilities between 0 and 1."}
    \hlkwd{stop}\hlstd{(errormsg)}
  \hlstd{\}}
  \hlstd{indicator1} \hlkwb{<-} \hlkwd{rep}\hlstd{(}\hlnum{1}\hlstd{,} \hlkwd{length}\hlstd{(q))}
  \hlstd{indicator1[q} \hlopt{!=} \hlnum{0}\hlstd{]} \hlkwb{<-} \hlnum{0} \hlcom{#indicator should be zero if x!=0}
  \hlstd{indicator2} \hlkwb{<-} \hlkwd{rep}\hlstd{(}\hlnum{1}\hlstd{,} \hlkwd{length}\hlstd{(q))}
  \hlstd{indicator2[q} \hlopt{<} \hlnum{1}\hlstd{]} \hlkwb{<-} \hlnum{0} \hlcom{#indicator should be zero if x<1}
  \hlstd{Fx} \hlkwb{<-} \hlstd{(}\hlnum{1}\hlopt{-}\hlstd{prob)} \hlopt{*} \hlstd{indicator1} \hlopt{+} \hlstd{indicator2}
  \hlkwd{return}\hlstd{(Fx)}
\hlstd{\}}
\hlstd{qbern} \hlkwb{<-} \hlkwa{function}\hlstd{(}\hlkwc{p}\hlstd{,} \hlkwc{prob}\hlstd{)\{}
  \hlkwa{if}\hlstd{(prob}\hlopt{<}\hlnum{0} \hlopt{|} \hlstd{prob}\hlopt{>}\hlnum{1}\hlstd{)\{}
    \hlstd{errormsg}\hlkwb{<-}\hlstr{"This function is only valid for success probabilities between 0 and 1."}
    \hlkwd{stop}\hlstd{(errormsg)}
  \hlstd{\}}
  \hlkwa{if}\hlstd{(}\hlkwd{any}\hlstd{(p}\hlopt{<}\hlnum{0}\hlstd{)} \hlopt{|} \hlkwd{any}\hlstd{(p}\hlopt{>}\hlnum{1}\hlstd{))\{}
    \hlstd{errormsg}\hlkwb{<-}\hlstr{"This function is only valid for percentiles between 0 and 1."}
    \hlkwd{stop}\hlstd{(errormsg)}
  \hlstd{\}}
  \hlstd{q} \hlkwb{<-} \hlkwd{rep}\hlstd{(}\hlnum{1}\hlstd{,} \hlkwd{length}\hlstd{(p))}
  \hlstd{q[p} \hlopt{<=} \hlnum{1}\hlopt{-}\hlstd{prob]} \hlkwb{<-} \hlnum{0} \hlcom{# should return zero if p<=(1-prob)}
  \hlstd{q[p} \hlopt{>} \hlnum{1}\hlopt{-}\hlstd{prob]} \hlkwb{<-} \hlnum{1}  \hlcom{# should return one if p>(1-prob)}
  \hlkwd{return}\hlstd{(q)}
\hlstd{\}}

\hlcom{## Plots for the PMF}
\hlstd{plotbern} \hlkwb{<-} \hlkwa{function}\hlstd{(}\hlkwc{prob}\hlstd{)\{} \hlcom{# Pass in the success probability}
  \hlstd{ggdat} \hlkwb{<-} \hlkwd{data.frame}\hlstd{(}\hlkwc{x} \hlstd{= (}\hlopt{-}\hlnum{1}\hlopt{:}\hlnum{2}\hlstd{),}
                      \hlkwc{f} \hlstd{=} \hlkwd{dbern}\hlstd{(}\hlkwc{x} \hlstd{= (}\hlopt{-}\hlnum{1}\hlopt{:}\hlnum{2}\hlstd{),} \hlkwc{prob} \hlstd{= prob),} \hlcom{#SETS X AXES NICELY}
                      \hlkwc{F} \hlstd{=} \hlkwd{pbern}\hlstd{(}\hlkwc{q} \hlstd{= (}\hlopt{-}\hlnum{1}\hlopt{:}\hlnum{2}\hlstd{),} \hlkwc{prob} \hlstd{= prob))}
  \hlcom{## Plot PMF}
  \hlstd{PMF} \hlkwb{<-} \hlkwd{ggplot}\hlstd{(}\hlkwc{data} \hlstd{= ggdat,} \hlkwd{aes}\hlstd{(}\hlkwc{x} \hlstd{= x))} \hlopt{+}
    \hlkwd{geom_linerange}\hlstd{(}\hlkwd{aes}\hlstd{(}\hlkwc{ymax} \hlstd{= f),} \hlkwc{ymin} \hlstd{=} \hlnum{0}\hlstd{)} \hlopt{+}
    \hlkwd{geom_hline}\hlstd{(}\hlkwc{yintercept} \hlstd{=} \hlnum{0}\hlstd{)} \hlopt{+}
    \hlkwd{theme_bw}\hlstd{()} \hlopt{+}
    \hlkwd{ylim}\hlstd{(}\hlnum{0}\hlstd{,} \hlnum{1}\hlstd{)} \hlopt{+}
    \hlkwd{xlab}\hlstd{(}\hlstr{"X"}\hlstd{)} \hlopt{+}
    \hlkwd{ylab}\hlstd{(}\hlkwd{bquote}\hlstd{(f[x](x)))} \hlopt{+}
    \hlkwd{ggtitle}\hlstd{(}\hlstr{"Bernoulli PMF"}\hlstd{,} \hlkwc{subtitle} \hlstd{=} \hlkwd{paste}\hlstd{(}\hlstr{"p ="}\hlstd{, prob))}

  \hlkwd{return}\hlstd{(PMF)}
\hlstd{\}}
\end{alltt}
\end{kframe}
\end{knitrout}
\begin{figure}
\centering
\begin{knitrout}
\definecolor{shadecolor}{rgb}{0.969, 0.969, 0.969}\color{fgcolor}\begin{kframe}
\begin{alltt}
\hlstd{(}\hlkwd{plotbern}\hlstd{(}\hlkwc{prob}\hlstd{=}\hlnum{0.25}\hlstd{))}\hlopt{/}
  \hlstd{(}\hlkwd{plotbern}\hlstd{(}\hlkwc{prob}\hlstd{=}\hlnum{0.50}\hlstd{))}\hlopt{/}\hlstd{(}\hlkwd{plotbern}\hlstd{(}\hlkwc{prob}\hlstd{=}\hlnum{0.75}\hlstd{))}
\end{alltt}
\end{kframe}
\includegraphics[width=\maxwidth]{figure/unnamed-chunk-2-1} 
\end{knitrout}
	\caption{Bernoulli Distribution PMFs for p=0.25, 0.5, and 0.75.}
	\label{BernPMF}
\end{figure}
	For all valid parameter values, the graph of the Bernoulli PMF shows two lines at the values 0 and 1, corresponding to the probabilities of those observations.  Thereby, the heights of these bars are controlled by the parameter p.  Lower parameter values correspond to a higher bar at X=0 and a lower bar at X=1, because the distribution represents a scenario in which it is more likely for a ``failure'' to occur (X=0) than a ``success" (X=1).  Higher parameter values correspond to a lower bar at X=0 and a higher bar at X=1, because the distribution represents a scenario in which it is more likely for a ``success'' to occur (X=1) than a ``failure" (X=0).  Three example PMFs of the Bernoulli distribution with varied parameter values are shown in figure \ref{BernPMF}.  I create these figures using \cite{ggplot}, \cite{tidyverse}, and \cite{patchwork}, and these are used throughout the homework to create plots.
	%%%%%%%%%%%%%%%%%%%%%%%%%%%%%%%%%%%%%%%%%%%%%%%%%%%%%%%%%%%%%%%%%%%%%%%%%%%%%%%%%%%%%%%%%%%
  %%%%%%%%%  Part (e)
  %%%%%%%%%%%%%%%%%%%%%%%%%%%%%%%%%%%%%%%%%%%%%%%%%%%%%%%%%%%%%%%%%%%%%%%%%%%%%%%%%%%%%%%%%%%
	 \item Graph the CDF for the same values of the parameter(s) as you did in 
	 Question \ref{q3PMF}. What does changing the parameter(s) do to the shape of 
	 the CDF? Comment on the aspects of the CDFs that show that the CDF is valid.\\
	 \textbf{Solution:}
\begin{knitrout}
\definecolor{shadecolor}{rgb}{0.969, 0.969, 0.969}\color{fgcolor}\begin{kframe}
\begin{alltt}
        \hlcom{##############################################################}
        \hlcom{##    ADAPTED FROM CHAPTER 6 NOTES CODE: ch6-Bernoulli.R}
        \hlcom{##############################################################}

\hlcom{## Plots for the CDF}
\hlstd{plotbern} \hlkwb{<-} \hlkwa{function}\hlstd{(}\hlkwc{prob}\hlstd{)\{} \hlcom{# Pass in the success probability}
  \hlstd{ggdat} \hlkwb{<-} \hlkwd{data.frame}\hlstd{(}\hlkwc{x} \hlstd{= (}\hlopt{-}\hlnum{1}\hlopt{:}\hlnum{2}\hlstd{),}
                      \hlkwc{f} \hlstd{=} \hlkwd{dbern}\hlstd{(}\hlkwc{x} \hlstd{= (}\hlopt{-}\hlnum{1}\hlopt{:}\hlnum{2}\hlstd{),} \hlkwc{prob} \hlstd{= prob),} \hlcom{#SETS X AXES NICELY}
                      \hlkwc{F} \hlstd{=} \hlkwd{pbern}\hlstd{(}\hlkwc{q} \hlstd{= (}\hlopt{-}\hlnum{1}\hlopt{:}\hlnum{2}\hlstd{),} \hlkwc{prob} \hlstd{= prob))}

  \hlstd{ggdat.openpoints} \hlkwb{<-} \hlkwd{data.frame}\hlstd{(}\hlkwc{x} \hlstd{= ggdat}\hlopt{$}\hlstd{x,}
                                 \hlkwc{y} \hlstd{=} \hlkwd{pbern}\hlstd{(ggdat}\hlopt{$}\hlstd{x}\hlopt{-}\hlnum{1}\hlstd{,} \hlkwc{prob} \hlstd{= prob))}
  \hlstd{ggdat.closedpoints} \hlkwb{<-} \hlkwd{data.frame}\hlstd{(}\hlkwc{x} \hlstd{= ggdat}\hlopt{$}\hlstd{x,}
                                   \hlkwc{y} \hlstd{=} \hlkwd{pbern}\hlstd{(ggdat}\hlopt{$}\hlstd{x,}\hlkwc{prob}\hlstd{=prob))}
  \hlstd{CDF}\hlkwb{<-}\hlkwd{ggplot}\hlstd{(}\hlkwc{data} \hlstd{= ggdat,} \hlkwd{aes}\hlstd{(}\hlkwc{x} \hlstd{= x,} \hlkwc{y} \hlstd{= F))} \hlopt{+}
    \hlkwd{geom_step}\hlstd{()}\hlopt{+}
    \hlkwd{geom_point}\hlstd{(}\hlkwc{data} \hlstd{= ggdat.openpoints,} \hlkwd{aes}\hlstd{(}\hlkwc{x} \hlstd{= x,} \hlkwc{y} \hlstd{= y),} \hlkwc{shape} \hlstd{=} \hlnum{1}\hlstd{)} \hlopt{+}
    \hlkwd{geom_point}\hlstd{(}\hlkwc{data} \hlstd{= ggdat.closedpoints,} \hlkwd{aes}\hlstd{(}\hlkwc{x} \hlstd{= x,} \hlkwc{y} \hlstd{= y))} \hlopt{+}
    \hlkwd{theme_bw}\hlstd{()}\hlopt{+}
    \hlkwd{xlab}\hlstd{(}\hlstr{"X"}\hlstd{)}\hlopt{+}
    \hlkwd{ylab}\hlstd{(}\hlkwd{bquote}\hlstd{(F[x](x)))}\hlopt{+}
    \hlkwd{ggtitle}\hlstd{(}\hlstr{"Bernoulli CDF"}\hlstd{,}\hlkwc{subtitle}\hlstd{=(}\hlkwd{paste}\hlstd{(}\hlstr{"p ="}\hlstd{, prob)))}
    \hlkwd{return}\hlstd{(CDF)}
\hlstd{\}}
\end{alltt}
\end{kframe}
\end{knitrout}
\begin{figure}
\centering
\begin{knitrout}
\definecolor{shadecolor}{rgb}{0.969, 0.969, 0.969}\color{fgcolor}\begin{kframe}
\begin{alltt}
\hlstd{(}\hlkwd{plotbern}\hlstd{(}\hlkwc{prob}\hlstd{=}\hlnum{0.25}\hlstd{))}\hlopt{/}\hlstd{(}\hlkwd{plotbern}\hlstd{(}\hlkwc{prob}\hlstd{=}\hlnum{0.50}\hlstd{))}\hlopt{/}\hlstd{(}\hlkwd{plotbern}\hlstd{(}\hlkwc{prob}\hlstd{=}\hlnum{0.75}\hlstd{))}
\end{alltt}
\end{kframe}
\includegraphics[width=\maxwidth]{figure/unnamed-chunk-4-1} 
\end{knitrout}
\caption{Bernoulli Distribution CDFs for p=0.25, 0.5, and 0.75.}
\label{BernCDF}
\end{figure}
  The changes in the CDF with varied parameter values are very similar to those of the PMF; because the CDF represents the cumulative sum of the PMF bars, before X=0, the CDF is 0.  The probability of observing an X lower than 0 is zero.  We can see that we have a valid CDF because at and after X=1, the CDF is 1. We are guaranteed to observe values only between 0 and 1 inclusive.  As with the PMF bars, lower parameter values correspond to a higher bar at X=0 (and therefore a lower bar at X=1 that takes the CDF to 1), because the distribution represents a scenario in which it is more likely for a ``failure'' to occur (X=0) than a ``success" (X=1).  Similarly again, higher parameter values correspond to a lower bar at X=0 (and therefore a higher bar at X=1 that takes the CDF to 1), because the distribution represents a scenario in which it is more likely for a ``success'' to occur (X=1) than a ``failure" (X=0).
  This results in a CDF that, visually, looks more ``concave up" for parameter values greater than 0.5, ``concave down" for values less than 0.5, and linear for values equal to 0.5.  Three example CDFs of the Bernoulli distribution with varied parameter values are shown in figure \ref{BernCDF}.
	%%%%%%%%%%%%%%%%%%%%%%%%%%%%%%%%%%%%%%%%%%%%%%%%%%%%%%%%%%%%%%%%%%%%%%%%%%%%%%%%%%%%%%%%%%%
  %%%%%%%%%  Part (f)
  %%%%%%%%%%%%%%%%%%%%%%%%%%%%%%%%%%%%%%%%%%%%%%%%%%%%%%%%%%%%%%%%%%%%%%%%%%%%%%%%%%%%%%%%%%%
  \item Generate a random sample of size $n=10, 25, 100$, and $1000$ for your 
  two sets of parameter(s). In a $4 \times 2$ grid, plot a histogram (with bin 
  size 1) of each set of data and superimpose the true mass function at the 
  specified parameter values. Interpret the results.\\
  \textbf{Solution:}
\begin{knitrout}
\definecolor{shadecolor}{rgb}{0.969, 0.969, 0.969}\color{fgcolor}\begin{kframe}
\begin{alltt}
\hlcom{##############################################################}
        \hlcom{##    ADAPTED FROM CHAPTER 6 NOTES CODE (pg176)}
        \hlcom{##############################################################}

\hlstd{rbern}\hlkwb{<-}\hlkwa{function}\hlstd{(}\hlkwc{n}\hlstd{,}\hlkwc{p}\hlstd{)\{}
  \hlstd{support}\hlkwb{<-}\hlkwd{c}\hlstd{(}\hlnum{0}\hlstd{,}\hlnum{1}\hlstd{)}
  \hlstd{x}\hlkwb{<-}\hlkwd{sample}\hlstd{(}\hlkwc{x}\hlstd{=support,} \hlkwc{size}\hlstd{=n,} \hlkwc{replace}\hlstd{=}\hlnum{TRUE}\hlstd{,} \hlkwc{prob}\hlstd{=}\hlkwd{dbern}\hlstd{(support,} \hlkwc{p}\hlstd{=p))}
  \hlkwd{return}\hlstd{(x)}
\hlstd{\}}

\hlstd{x.10.50}   \hlkwb{<-}\hlkwd{data.frame}\hlstd{(}\hlkwd{rbern}\hlstd{(}\hlnum{10}\hlstd{,} \hlnum{.5}\hlstd{))}
\hlstd{x.25.50}   \hlkwb{<-}\hlkwd{data.frame}\hlstd{(}\hlkwd{rbern}\hlstd{(}\hlnum{25}\hlstd{,} \hlnum{.5}\hlstd{))}
\hlstd{x.100.50}  \hlkwb{<-}\hlkwd{data.frame}\hlstd{(}\hlkwd{rbern}\hlstd{(}\hlnum{100}\hlstd{,} \hlnum{.5}\hlstd{))}
\hlstd{x.1000.50} \hlkwb{<-}\hlkwd{data.frame}\hlstd{(}\hlkwd{rbern}\hlstd{(}\hlnum{1000}\hlstd{,} \hlnum{.5}\hlstd{))}

\hlstd{x.10.75}   \hlkwb{<-}\hlkwd{data.frame}\hlstd{(}\hlkwd{rbern}\hlstd{(}\hlnum{10}\hlstd{,} \hlnum{.75}\hlstd{))}
\hlstd{x.25.75}   \hlkwb{<-}\hlkwd{data.frame}\hlstd{(}\hlkwd{rbern}\hlstd{(}\hlnum{25}\hlstd{,} \hlnum{.75}\hlstd{))}
\hlstd{x.100.75}  \hlkwb{<-}\hlkwd{data.frame}\hlstd{(}\hlkwd{rbern}\hlstd{(}\hlnum{100}\hlstd{,} \hlnum{.75}\hlstd{))}
\hlstd{x.1000.75} \hlkwb{<-}\hlkwd{data.frame}\hlstd{(}\hlkwd{rbern}\hlstd{(}\hlnum{1000}\hlstd{,} \hlnum{.75}\hlstd{))}
\hlcom{## PMF for p=0.5 }
\hlstd{ggdat} \hlkwb{<-} \hlkwd{data.frame}\hlstd{(}\hlkwc{x} \hlstd{= (}\hlopt{-}\hlnum{1}\hlopt{:}\hlnum{2}\hlstd{),}
                      \hlkwc{f} \hlstd{=} \hlkwd{dbern}\hlstd{(}\hlkwc{x} \hlstd{= (}\hlopt{-}\hlnum{1}\hlopt{:}\hlnum{2}\hlstd{),} \hlkwc{prob} \hlstd{=} \hlnum{0.5}\hlstd{),}
                      \hlkwc{F} \hlstd{=} \hlkwd{pbern}\hlstd{(}\hlkwc{q} \hlstd{= (}\hlopt{-}\hlnum{1}\hlopt{:}\hlnum{2}\hlstd{),} \hlkwc{prob} \hlstd{=} \hlnum{0.5}\hlstd{))}

\hlstd{g1}\hlkwb{<-}\hlkwd{ggplot}\hlstd{(}\hlkwa{NULL}\hlstd{)} \hlopt{+}                                      \hlcom{#left NULL so that we can superimpose}
  \hlkwd{geom_histogram}\hlstd{(}\hlkwc{data}\hlstd{=x.10.50,}                          \hlcom{#uses random sample}
                 \hlkwd{aes}\hlstd{(}\hlkwc{x}\hlstd{=rbern.10..0.5.),}                 \hlcom{#makes histogram}
                 \hlkwc{breaks}\hlstd{=}\hlkwd{seq}\hlstd{(}\hlopt{-}\hlnum{0.5}\hlstd{,}\hlnum{1.5}\hlstd{,}\hlnum{1}\hlstd{),}                \hlcom{#Set up bins to use}
                 \hlkwc{fill} \hlstd{=} \hlstr{"lightgrey"}\hlstd{,} \hlkwc{color}\hlstd{=}\hlstr{"black"}\hlstd{)} \hlopt{+}   \hlcom{#color the histogram}
  \hlkwd{xlab}\hlstd{(}\hlstr{"X"}\hlstd{)} \hlopt{+}                                           \hlcom{#x axis label}
  \hlkwd{ylab}\hlstd{(}\hlstr{"Frequency"}\hlstd{)}\hlopt{+}                                    \hlcom{#y axis label}
  \hlkwd{ggtitle}\hlstd{(}\hlstr{"Random Bernoulli Distribution"}\hlstd{,}
          \hlkwc{subtitle} \hlstd{=} \hlstr{"With p=0.5 and n=10"}\hlstd{)} \hlopt{+}           \hlcom{#add title to plot}
  \hlkwd{geom_hline}\hlstd{(}\hlkwc{yintercept}\hlstd{=}\hlnum{0}\hlstd{)} \hlopt{+}                            \hlcom{#adds a line for the x-axis}
  \hlkwd{geom_linerange}\hlstd{(}\hlkwc{data}\hlstd{=ggdat,}                            \hlcom{#uses earlier PMF data}
                 \hlkwd{aes}\hlstd{(}\hlkwc{x}\hlstd{=x,} \hlkwc{ymax}\hlstd{=}\hlnum{10}\hlopt{*}\hlstd{f,} \hlkwc{ymin}\hlstd{=}\hlnum{0}\hlstd{),}           \hlcom{#Adds PMF (weighted by number of observations)}
                 \hlkwc{color}\hlstd{=}\hlstr{"red"}\hlstd{)}                           \hlcom{#Makes PMF red}

\hlstd{g2}\hlkwb{<-}\hlkwd{ggplot}\hlstd{(}\hlkwa{NULL}\hlstd{)} \hlopt{+}                                      \hlcom{#left NULL so that we can superimpose}
  \hlkwd{geom_histogram}\hlstd{(}\hlkwc{data}\hlstd{=x.25.50,}                          \hlcom{#uses random sample}
                 \hlkwd{aes}\hlstd{(}\hlkwc{x}\hlstd{=rbern.25..0.5.),}                 \hlcom{#makes histogram}
                 \hlkwc{breaks}\hlstd{=}\hlkwd{seq}\hlstd{(}\hlopt{-}\hlnum{0.5}\hlstd{,}\hlnum{1.5}\hlstd{,}\hlnum{1}\hlstd{),}                \hlcom{#Set up bins to use}
                 \hlkwc{fill} \hlstd{=} \hlstr{"lightgrey"}\hlstd{,} \hlkwc{color}\hlstd{=}\hlstr{"black"}\hlstd{)} \hlopt{+}   \hlcom{#color the histogram}
  \hlkwd{xlab}\hlstd{(}\hlstr{"X"}\hlstd{)} \hlopt{+}                                           \hlcom{#x axis label}
  \hlkwd{ylab}\hlstd{(}\hlstr{"Frequency"}\hlstd{)}\hlopt{+}                                    \hlcom{#y axis label}
  \hlkwd{ggtitle}\hlstd{(}\hlstr{"Random Bernoulli Distribution"}\hlstd{,}
          \hlkwc{subtitle} \hlstd{=} \hlstr{"With p=0.5 and n=25"}\hlstd{)} \hlopt{+}           \hlcom{#add title to plot}
  \hlkwd{geom_hline}\hlstd{(}\hlkwc{yintercept}\hlstd{=}\hlnum{0}\hlstd{)} \hlopt{+}                            \hlcom{#adds a line for the x-axis}
  \hlkwd{geom_linerange}\hlstd{(}\hlkwc{data}\hlstd{=ggdat,}                            \hlcom{#uses earlier PMF data}
                 \hlkwd{aes}\hlstd{(}\hlkwc{x}\hlstd{=x,} \hlkwc{ymax}\hlstd{=}\hlnum{25}\hlopt{*}\hlstd{f,} \hlkwc{ymin}\hlstd{=}\hlnum{0}\hlstd{),}           \hlcom{#Adds PMF (weighted by number of observations)}
                 \hlkwc{color}\hlstd{=}\hlstr{"red"}\hlstd{)}                           \hlcom{#Makes PMF red}

\hlstd{g3}\hlkwb{<-}\hlkwd{ggplot}\hlstd{(}\hlkwa{NULL}\hlstd{)} \hlopt{+}                                      \hlcom{#left NULL so that we can superimpose}
  \hlkwd{geom_histogram}\hlstd{(}\hlkwc{data}\hlstd{=x.100.50,}                         \hlcom{#uses random sample}
                 \hlkwd{aes}\hlstd{(}\hlkwc{x}\hlstd{=rbern.100..0.5.),}                \hlcom{#makes histogram}
                 \hlkwc{breaks}\hlstd{=}\hlkwd{seq}\hlstd{(}\hlopt{-}\hlnum{0.5}\hlstd{,}\hlnum{1.5}\hlstd{,}\hlnum{1}\hlstd{),}                \hlcom{#Set up bins to use}
                 \hlkwc{fill} \hlstd{=} \hlstr{"lightgrey"}\hlstd{,} \hlkwc{color}\hlstd{=}\hlstr{"black"}\hlstd{)} \hlopt{+}   \hlcom{#color the histogram}
  \hlkwd{xlab}\hlstd{(}\hlstr{"X"}\hlstd{)} \hlopt{+}                                           \hlcom{#x axis label}
  \hlkwd{ylab}\hlstd{(}\hlstr{"Frequency"}\hlstd{)}\hlopt{+}                                    \hlcom{#y axis label}
  \hlkwd{ggtitle}\hlstd{(}\hlstr{"Random Bernoulli Distribution"}\hlstd{,}
          \hlkwc{subtitle} \hlstd{=} \hlstr{"With p=0.5 and n=100"}\hlstd{)} \hlopt{+}          \hlcom{#add title to plot}
  \hlkwd{geom_hline}\hlstd{(}\hlkwc{yintercept}\hlstd{=}\hlnum{0}\hlstd{)} \hlopt{+}                            \hlcom{#adds a line for the x-axis}
  \hlkwd{geom_linerange}\hlstd{(}\hlkwc{data}\hlstd{=ggdat,}                            \hlcom{#uses earlier PMF data}
                 \hlkwd{aes}\hlstd{(}\hlkwc{x}\hlstd{=x,} \hlkwc{ymax}\hlstd{=}\hlnum{100}\hlopt{*}\hlstd{f,} \hlkwc{ymin}\hlstd{=}\hlnum{0}\hlstd{),}          \hlcom{#Adds PMF (weighted by number of observations)}
                 \hlkwc{color}\hlstd{=}\hlstr{"red"}\hlstd{)}                           \hlcom{#Makes PMF red}


\hlstd{g4}\hlkwb{<-}\hlkwd{ggplot}\hlstd{(}\hlkwa{NULL}\hlstd{)} \hlopt{+}                                      \hlcom{#left NULL so that we can superimpose}
  \hlkwd{geom_histogram}\hlstd{(}\hlkwc{data}\hlstd{=x.1000.50,}                        \hlcom{#uses random sample}
                 \hlkwd{aes}\hlstd{(}\hlkwc{x}\hlstd{=rbern.1000..0.5.),}               \hlcom{#makes histogram}
                 \hlkwc{breaks}\hlstd{=}\hlkwd{seq}\hlstd{(}\hlopt{-}\hlnum{0.5}\hlstd{,}\hlnum{1.5}\hlstd{,}\hlnum{1}\hlstd{),}                \hlcom{#Set up bins to use}
                 \hlkwc{fill} \hlstd{=} \hlstr{"lightgrey"}\hlstd{,} \hlkwc{color}\hlstd{=}\hlstr{"black"}\hlstd{)} \hlopt{+}   \hlcom{#color the histogram}
  \hlkwd{xlab}\hlstd{(}\hlstr{"X"}\hlstd{)} \hlopt{+}                                           \hlcom{#x axis label}
  \hlkwd{ylab}\hlstd{(}\hlstr{"Frequency"}\hlstd{)}\hlopt{+}                                    \hlcom{#y axis label}
  \hlkwd{ggtitle}\hlstd{(}\hlstr{"Random Bernoulli Distribution"}\hlstd{,}
          \hlkwc{subtitle} \hlstd{=} \hlstr{"With p=0.5 and n=1000"}\hlstd{)} \hlopt{+}         \hlcom{#add title to plot}
  \hlkwd{geom_hline}\hlstd{(}\hlkwc{yintercept}\hlstd{=}\hlnum{0}\hlstd{)} \hlopt{+}                            \hlcom{#adds a line for the x-axis}
  \hlkwd{geom_linerange}\hlstd{(}\hlkwc{data}\hlstd{=ggdat,}                            \hlcom{#uses earlier PMF data}
                 \hlkwd{aes}\hlstd{(}\hlkwc{x}\hlstd{=x,} \hlkwc{ymax}\hlstd{=}\hlnum{1000}\hlopt{*}\hlstd{f,} \hlkwc{ymin}\hlstd{=}\hlnum{0}\hlstd{),}         \hlcom{#Adds PMF (weighted by number of observations)}
                 \hlkwc{color}\hlstd{=}\hlstr{"red"}\hlstd{)}                           \hlcom{#Makes PMF red}

\hlcom{## PMF for p=0.5 }
\hlstd{ggdat} \hlkwb{<-} \hlkwd{data.frame}\hlstd{(}\hlkwc{x} \hlstd{= (}\hlopt{-}\hlnum{1}\hlopt{:}\hlnum{2}\hlstd{),}
                      \hlkwc{f} \hlstd{=} \hlkwd{dbern}\hlstd{(}\hlkwc{x} \hlstd{= (}\hlopt{-}\hlnum{1}\hlopt{:}\hlnum{2}\hlstd{),} \hlkwc{prob} \hlstd{=} \hlnum{0.75}\hlstd{),}
                      \hlkwc{F} \hlstd{=} \hlkwd{pbern}\hlstd{(}\hlkwc{q} \hlstd{= (}\hlopt{-}\hlnum{1}\hlopt{:}\hlnum{2}\hlstd{),} \hlkwc{prob} \hlstd{=} \hlnum{0.75}\hlstd{))}

\hlstd{g5}\hlkwb{<-}\hlkwd{ggplot}\hlstd{(}\hlkwa{NULL}\hlstd{)} \hlopt{+}                                      \hlcom{#left NULL so that we can superimpose}
  \hlkwd{geom_histogram}\hlstd{(}\hlkwc{data}\hlstd{=x.10.75,}                          \hlcom{#uses random sample}
                 \hlkwd{aes}\hlstd{(}\hlkwc{x}\hlstd{=rbern.10..0.75.),}                \hlcom{#makes histogram}
                 \hlkwc{breaks}\hlstd{=}\hlkwd{seq}\hlstd{(}\hlopt{-}\hlnum{0.5}\hlstd{,}\hlnum{1.5}\hlstd{,}\hlnum{1}\hlstd{),}                \hlcom{#Set up bins to use}
                 \hlkwc{fill} \hlstd{=} \hlstr{"lightgrey"}\hlstd{,} \hlkwc{color}\hlstd{=}\hlstr{"black"}\hlstd{)} \hlopt{+}   \hlcom{#color the histogram}
  \hlkwd{xlab}\hlstd{(}\hlstr{"X"}\hlstd{)} \hlopt{+}                                           \hlcom{#x axis label}
  \hlkwd{ylab}\hlstd{(}\hlstr{"Frequency"}\hlstd{)}\hlopt{+}                                    \hlcom{#y axis label}
  \hlkwd{ggtitle}\hlstd{(}\hlstr{"Random Bernoulli Distribution"}\hlstd{,}
          \hlkwc{subtitle} \hlstd{=} \hlstr{"With p=0.75 and n=10"}\hlstd{)} \hlopt{+}          \hlcom{#add title to plot}
  \hlkwd{geom_hline}\hlstd{(}\hlkwc{yintercept}\hlstd{=}\hlnum{0}\hlstd{)} \hlopt{+}                            \hlcom{#adds a line for the x-axis}
  \hlkwd{geom_linerange}\hlstd{(}\hlkwc{data}\hlstd{=ggdat,}                            \hlcom{#uses earlier PMF data}
                 \hlkwd{aes}\hlstd{(}\hlkwc{x}\hlstd{=x,} \hlkwc{ymax}\hlstd{=}\hlnum{10}\hlopt{*}\hlstd{f,} \hlkwc{ymin}\hlstd{=}\hlnum{0}\hlstd{),}           \hlcom{#Adds PMF (weighted by number of observations)}
                 \hlkwc{color}\hlstd{=}\hlstr{"red"}\hlstd{)}                           \hlcom{#Makes PMF red}

\hlstd{g6}\hlkwb{<-}\hlkwd{ggplot}\hlstd{(}\hlkwa{NULL}\hlstd{)} \hlopt{+}                                      \hlcom{#left NULL so that we can superimpose}
  \hlkwd{geom_histogram}\hlstd{(}\hlkwc{data}\hlstd{=x.25.75,}                          \hlcom{#uses random sample}
                 \hlkwd{aes}\hlstd{(}\hlkwc{x}\hlstd{=rbern.25..0.75.),}                \hlcom{#makes histogram}
                 \hlkwc{breaks}\hlstd{=}\hlkwd{seq}\hlstd{(}\hlopt{-}\hlnum{0.5}\hlstd{,}\hlnum{1.5}\hlstd{,}\hlnum{1}\hlstd{),}                \hlcom{#Set up bins to use}
                 \hlkwc{fill} \hlstd{=} \hlstr{"lightgrey"}\hlstd{,} \hlkwc{color}\hlstd{=}\hlstr{"black"}\hlstd{)} \hlopt{+}   \hlcom{#color the histogram}
  \hlkwd{xlab}\hlstd{(}\hlstr{"X"}\hlstd{)} \hlopt{+}                                           \hlcom{#x axis label}
  \hlkwd{ylab}\hlstd{(}\hlstr{"Frequency"}\hlstd{)}\hlopt{+}                                    \hlcom{#y axis label}
  \hlkwd{ggtitle}\hlstd{(}\hlstr{"Random Bernoulli Distribution"}\hlstd{,}
          \hlkwc{subtitle} \hlstd{=} \hlstr{"With p=0.75 and n=25"}\hlstd{)} \hlopt{+}          \hlcom{#add title to plot}
  \hlkwd{geom_hline}\hlstd{(}\hlkwc{yintercept}\hlstd{=}\hlnum{0}\hlstd{)} \hlopt{+}                            \hlcom{#adds a line for the x-axis}
  \hlkwd{geom_linerange}\hlstd{(}\hlkwc{data}\hlstd{=ggdat,}                            \hlcom{#uses earlier PMF data}
                 \hlkwd{aes}\hlstd{(}\hlkwc{x}\hlstd{=x,} \hlkwc{ymax}\hlstd{=}\hlnum{25}\hlopt{*}\hlstd{f,} \hlkwc{ymin}\hlstd{=}\hlnum{0}\hlstd{),}           \hlcom{#Adds PMF (weighted by number of observations)}
                 \hlkwc{color}\hlstd{=}\hlstr{"red"}\hlstd{)}                           \hlcom{#Makes PMF red}

\hlstd{g7}\hlkwb{<-}\hlkwd{ggplot}\hlstd{(}\hlkwa{NULL}\hlstd{)} \hlopt{+}                                      \hlcom{#left NULL so that we can superimpose}
  \hlkwd{geom_histogram}\hlstd{(}\hlkwc{data}\hlstd{=x.100.75,}                         \hlcom{#uses random sample}
                 \hlkwd{aes}\hlstd{(}\hlkwc{x}\hlstd{=rbern.100..0.75.),}               \hlcom{#makes histogram}
                 \hlkwc{breaks}\hlstd{=}\hlkwd{seq}\hlstd{(}\hlopt{-}\hlnum{0.5}\hlstd{,}\hlnum{1.5}\hlstd{,}\hlnum{1}\hlstd{),}                \hlcom{#Set up bins to use}
                 \hlkwc{fill} \hlstd{=} \hlstr{"lightgrey"}\hlstd{,} \hlkwc{color}\hlstd{=}\hlstr{"black"}\hlstd{)} \hlopt{+}   \hlcom{#color the histogram}
  \hlkwd{xlab}\hlstd{(}\hlstr{"X"}\hlstd{)} \hlopt{+}                                           \hlcom{#x axis label}
  \hlkwd{ylab}\hlstd{(}\hlstr{"Frequency"}\hlstd{)}\hlopt{+}                                    \hlcom{#y axis label}
  \hlkwd{ggtitle}\hlstd{(}\hlstr{"Random Bernoulli Distribution"}\hlstd{,}
          \hlkwc{subtitle} \hlstd{=} \hlstr{"With p=0.75 and n=100"}\hlstd{)} \hlopt{+}         \hlcom{#add title to plot}
  \hlkwd{geom_hline}\hlstd{(}\hlkwc{yintercept}\hlstd{=}\hlnum{0}\hlstd{)} \hlopt{+}                            \hlcom{#adds a line for the x-axis}
  \hlkwd{geom_linerange}\hlstd{(}\hlkwc{data}\hlstd{=ggdat,}                            \hlcom{#uses earlier PMF data}
                 \hlkwd{aes}\hlstd{(}\hlkwc{x}\hlstd{=x,} \hlkwc{ymax}\hlstd{=}\hlnum{100}\hlopt{*}\hlstd{f,} \hlkwc{ymin}\hlstd{=}\hlnum{0}\hlstd{),}          \hlcom{#Adds PMF (weighted by number of observations)}
                 \hlkwc{color}\hlstd{=}\hlstr{"red"}\hlstd{)}                           \hlcom{#Makes PMF red}


\hlstd{g8}\hlkwb{<-}\hlkwd{ggplot}\hlstd{(}\hlkwa{NULL}\hlstd{)} \hlopt{+}                                      \hlcom{#left NULL so that we can superimpose}
  \hlkwd{geom_histogram}\hlstd{(}\hlkwc{data}\hlstd{=x.1000.75,}                        \hlcom{#uses random sample}
                 \hlkwd{aes}\hlstd{(}\hlkwc{x}\hlstd{=rbern.1000..0.75.),}               \hlcom{#makes histogram}
                 \hlkwc{breaks}\hlstd{=}\hlkwd{seq}\hlstd{(}\hlopt{-}\hlnum{0.5}\hlstd{,}\hlnum{1.5}\hlstd{,}\hlnum{1}\hlstd{),}                \hlcom{#Set up bins to use}
                 \hlkwc{fill} \hlstd{=} \hlstr{"lightgrey"}\hlstd{,} \hlkwc{color}\hlstd{=}\hlstr{"black"}\hlstd{)} \hlopt{+}   \hlcom{#color the histogram}
  \hlkwd{xlab}\hlstd{(}\hlstr{"X"}\hlstd{)} \hlopt{+}                                           \hlcom{#x axis label}
  \hlkwd{ylab}\hlstd{(}\hlstr{"Frequency"}\hlstd{)}\hlopt{+}                                    \hlcom{#y axis label}
  \hlkwd{ggtitle}\hlstd{(}\hlstr{"Random Bernoulli Distribution"}\hlstd{,}
          \hlkwc{subtitle} \hlstd{=} \hlstr{"With p=0.75 and n=1000"}\hlstd{)} \hlopt{+}        \hlcom{#add title to plot}
  \hlkwd{geom_hline}\hlstd{(}\hlkwc{yintercept}\hlstd{=}\hlnum{0}\hlstd{)} \hlopt{+}                            \hlcom{#adds a line for the x-axis}
  \hlkwd{geom_linerange}\hlstd{(}\hlkwc{data}\hlstd{=ggdat,}                            \hlcom{#uses earlier PMF data}
                 \hlkwd{aes}\hlstd{(}\hlkwc{x}\hlstd{=x,} \hlkwc{ymax}\hlstd{=}\hlnum{1000}\hlopt{*}\hlstd{f,} \hlkwc{ymin}\hlstd{=}\hlnum{0}\hlstd{),}         \hlcom{#Adds PMF (weighted by number of observations)}
                 \hlkwc{color}\hlstd{=}\hlstr{"red"}\hlstd{)}                           \hlcom{#Makes PMF red}
\end{alltt}
\end{kframe}
\end{knitrout}
\begin{figure}[H]
\centering
\begin{knitrout}
\definecolor{shadecolor}{rgb}{0.969, 0.969, 0.969}\color{fgcolor}\begin{kframe}
\begin{alltt}
\hlstd{(g1} \hlopt{+} \hlstd{g5)} \hlopt{/} \hlstd{(g2} \hlopt{+} \hlstd{g6)} \hlopt{/} \hlstd{(g3} \hlopt{+} \hlstd{g7)} \hlopt{/} \hlstd{(g4} \hlopt{+} \hlstd{g8)}
\end{alltt}
\end{kframe}
\includegraphics[width=\maxwidth]{figure/unnamed-chunk-6-1} 
\end{knitrout}
\caption{Bernoulli Distributionsfor randomly generated samples, with varied n and p.  The red lines prepresent the PMF, while the histograms show the observed counts.}
\label{BernRand}
\end{figure}
  The greater the size of our random samples, the closer, generally, the spread of the observations we make are to the number we would predict from the PMF.  We can see this in that the heights of the grey histogram bars seem to get closer to the red lines that represent the PMF's prediction.
\end{enumerate}
\bibliography{bib}
\end{document}
